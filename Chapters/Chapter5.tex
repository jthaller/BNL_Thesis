% Future Work
One-shot learning is arguably the most challenging problem in machine learning \cite{transferlearning3}, and continued work is required to improve the model's ability to predict disorder in experimental spectra. As stage two and stage three of the transfer learning process are conducted, it will be critical to ensure no data leakage is committed. The transfer learning process described in Chapter 4 depends heavily on segmenting the data (both original and augmented) into different learning stages. Any data leakage between the stages would inject more bias the model and the final predictions on experimental data. While more time should be dedicated to tuning the hyperparameters and experimenting with different architectures, the approach taken in this thesis---as described in Chapter 4---is by no means the only technique to apply transfer learning. Recent work \cite{meta-learning-orig} \cite{huawei-meta-sgd} \cite{meta-learning2018} on meta-learning, the process by which a model learns to solve many sub-problems and is optimized for learning new problems, offers a promising direction that may be pursued. Alternatively, there has been interesting work on few-shot learning for regression problems involving deep embeddings \cite{deepembeddings-few-shot}.

While current limitations in the quantity of experimental data stymie the efforts to complete a model for predicting the mean squared disorder of experimental bulk gold XANES spectra, we present reasonable evidence that bond-length disorder is encoded within the spectra. Further work on transfer learning, improved simulation quality, or a much larger corpus of experimental data may yield the desired predictive capabilities. A successful neural network capable of predicting bulk gold should, in theory, be able to learn to predict disorder from other structures via a similar transfer learning process as described in Chapter 4.

Even with successful predictions on the original experimental spectra, more evidence is required to test the network's predictive validity. This evidence should be in the form of predicting the MSD from new, experimental spectra. It is important that the model has never seen these spectra, nor a data-augmented version of these spectra. Even though the model is trained only on a simulation and data-augmented experimental spectra, there is a possibility that the network is biased beyond what we expect (through transfer learning model selection). Thus, correctly predicting the mean squared displacement from novel Au XANES spectra would build confidence that our model has learned---rather than memorized---the encoded disorder within XANES spectra.