% Another mock chapter

Here is a second mock chapter.  As far as the \LaTeX ~is concerned, it is in no way different from the introduction excepting that it appears after it in the main .tex file.  As before, it can be populated with sections, subsections, figures, etc. as you see fit.

In fact, you will probably write perhaps three to six chapters for your thesis depending on how your work is most effectively organized.  Most theses will contain an introduction, at least one `body' chapter, and some sort of conclusions/future directions chapter.  Most theses will also include an appendix or two \ldots

\section{Autoencoders}
Talk about how autoencoders work. Give a nice broad explanation and really go into the math. Include some nice diagrams

Here's \cite{ng2011sparse} a good source to read and model off of. Here \cite{Bhowick2019} is another paper that might be interesting to read. It's about getting noise free data from the original data using an autoencoder. Neat idea, and could actually be very relevant because they're using geophysical data.