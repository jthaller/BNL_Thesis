% % Chapter 2: XAFS Simulations
% % Chapter 2: XANES
% \textit{In this section, I can write about XANES to a super in-depth extent, and likely the bulk of this chapter will be about FEFF and FDMNES theoretical calculations}
% \section{Maybe derivation of EXAFS Equation}
% \section{FEFF vs. FDMNES Approaches}
% \section{Theoretical XANES Calculations}
% -------------------------------------------------------------

In order for a neural network to be able to make any predictions, it must first be trained on a large quantity of data. Specifically, in order to teach our neural network to predict the mean sqared displacement (MSD)\todo[color=yellow!40]{Of just Au NPs? Can we generalize this to sigma of the PRDF?}, we must first generate a large quanity of training data comprised of XANES spectra, each labeled with a known MSD. Rather than spend countless hours and money generating this training data experimentally or simulating each possible disordered strucuture individually, we have generated this training data via clever statistical averging of simulations data from simple, non-disordered structures. In this chapter, we explain this statistical process in-depth -- from creating the dataset for initial simulations to finally creating the neural network training data.
\todo[inline, color=green!40]{Maybe give a brief outline of the process so you can understand the point of the sections as a walk-through before the very end}

\section{Generating Distortion Not Disorder}
Instead of creating structures with a range of disorder, we begin by creating structures with distortion instead, where distortion is defined here as a radial shift in all atomic positions away from (or towards) the center atomic absorber.