% Here is where you'll include the actual text for your abstract.  Because this lives inside the main document and will be included with a \input command, no tags are needed to start a new document.  Instead, just type.


% Your abstract will summarize your thesis in one or two paragraphs.  This brief summary should emphasize methods and results, not introductory material.

% X-ray absorption spectroscopy (XAS) is a widely used experimental technique for material characterization. The two
A recently published method \cite{Timoshenko2017} enables the decoding of X-ray absorption near edge structure (XANES) spectra of nanoparticles to obtain important structural descriptors: coordination numbers and bond distances. Utilizing supervised machine learning (ML), the method trains an artificial neural network (ANN) to recognize a relationship between the nanoparticle structure and the XANES spectrum. Once trained, the ANN is used to ``invert'' an unknown spectrum to obtain the corresponding descriptors of the catalyst structure. Bond strain is known to be an important catalytic descriptor, yet, its accurate determination in reaction conditions is hampered by high temperature and low weight loading of real catalysts. ML–assisted XANES analysis offers a promising new direction for extracting the bond strain information from XANES---and not from extended x-ray absorption fine structure (EXAFS) analysis. Using simulated XANES spectra of Au nanoparticles, we have developed an ANN capable of ``inverting'' an unseen XANES spectrum and predicting structural disorder in the form of mean-squared displacement. The utility of the method was demonstrated on both the computer-simulated nanoparticles of different sizes and degrees of disorder, as well as on experimental data of disordered nanoparticles.

% Further, the network is being trained to predict additional disorder descriptors, which would allow for the reconstruction of the radial distribution function, $g(r)$. This prediction is possible in part due to a new statistical averaging approach whereby XANES spectra of disordered structures are created from simulated XANES spectra of non-disordered structures.

% https://www.worldscientific.com/doi/abs/10.1142/9789811204579_0007
% Y. Lin, M. Topsakal, J. Timoshenko, D. Lu, S. Yoo, A. I. Frenkel
% Machine-Learning assisted structure determination of metallic nanoparticles: A benchmark
% In: Handbook on Big Data and Machine Learning in the Physical Sciences, World Scientific Series on Emerging Technologies, v.2, Chapter 7, pp. 127-140 (2020).