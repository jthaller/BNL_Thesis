% Look!  A mock introduction

The introduction is one of the most important pieces of your thesis.  Here is a place for you to introduce the problem(s) on which you have worked and place them in the larger context of your field.  You should aim to ensure that this section is completely understandable to virtually anyone - and certainly anyone with a sophomore-level grasp of physics.  Presumably this will include references to the literature.

In addition to setting your work into context, a second good idea for your introduction is to give a short outline for what the rest of your thesis will discuss.  This is often done in the closing paragraph(s) of the introduction with sentences like ``In the following chapters \ldots " and ``Chapter 2 discusses \ldots"  Tremendous detail is not required in this outline, but rather just a brief road map for the rest of the document.

\section{Traditional XANES}

The \texttt{\textbackslash section} tag will create a new section within a chapter.  Sections will be sequenced with digits following a decimal point in the table of contents, i.e. this is section 1.1.


\section{Machine Learning in Science}


Probably want to talk about these papers in this section \cite{timoshenko2018neural} \cite{Timoshenko2017}.

